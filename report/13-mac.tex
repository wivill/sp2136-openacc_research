\chapter{Ajustes para equipos Apple}
Algunos usuarios del clúster tienen como plataforma una computadora Macintosh o similar, la cual en esencia está basada en Unix, aunque no sea necesariamente POSIX compliant. Estos equipos ofrecen una experiencia bastante amena siempre y cuando se realicen pequeños ajustes al equipo. Esta sección documenta los procedimientos a seguir para alcanzar una óptima experiencia de uso para los usuarios de Mac OS.

\section{Redirección de sesión X11 para poder utilizar aplicaciones gráficas}

Para poder utilizar aplicaciones con interfaz gráfica, es necesario editar los archivos de configuración de ssh en el equipo local. Para ello debemos abrir una terminal, la cual es fácilmente ubicable utilizando el buscador incorporado al lado derecho de la pantalla. Una vez ahí, hacemos lo siguiente:

\begin{lstlisting}
sudo nano /etc/ssh/sshd_config
\end{lstlisting}

Una vez en el archivo, agregamos las siguientes líneas al final del archivo y salvamos:

\begin{lstlisting}
X11Forwarding yes
X11DisplayOffset 10
\end{lstlisting}

Posteriormente editamos el siguiente archivo:

\begin{lstlisting}
sudo nano /etc/ssh/ssh_config
\end{lstlisting}

Nos dirigimos bajo la sección Host * y nos aseguramos de que contenga lo siguiente:

\begin{lstlisting}
 Host *
#   SendEnv LANG LC_*
#   ForwardAgent no
   ForwardX11 yes
   XAuthLocation /usr/X11/bin/xauth
   ServerAliveInterval 60
   ForwardX11Timeout 596h
\end{lstlisting}

Nuevamente salvamos el archivo, y ahora procedemos a reiniciar el servicio de ssh:

\begin{lstlisting}
sudo launchctl unload /System/Library/LaunchDaemons/ssh.plist
sudo launchctl load -w /System/Library/LaunchDaemons/ssh.plist
\end{lstlisting}

Una vez hecho lo anterior, procedemos a instalar una versión opensource del servicio X11, denominado xquartz. Este software es posible descargarlo del sitio \url{https://www.xquartz.org}. Se descomprime y se ejecuta el instalador. Una vez finalizado, se debe cerrar y volver a abrir la sesión actual para que el cambio de servidor X11 tenga efecto. Finalmente, basta con iniciar sesión por ssh con las opciones por defecto:

\begin{lstlisting}
ssh -X usuario@hostname
ssh -Y usuario@hostname
\end{lstlisting}

\section{iTerm: Emulador de terminal equivalente a Terminator de Linux}


\clearpage