\chapter{Introducción}
El Colaboratorio Nacional de Computación Avanzada (CNCA) es un espacio compartido y distribuido en el que interactúan investigadores y desarrolladores de computación avanzada, es decir, de las áreas de convergencia y aplicación de las ciencias de la computación en las ciencias naturales, ingenierías, humanidades, ciencias sociales y artes. Fue concebido como un espacio de colaboración interdisciplinario y multisectorial. 

Entre los recursos con los que se cuenta se encuentra el clúster, denominado Cadejos, el cual está a disposición de las universidades integrantes del Consejo Nacional de Rectores (CONARE) para facilitar el desarrollo de tareas computacionales complejas a quienes realizan investigación y a estudiantes cuyos proyectos requieren de recursos computacionales considerables.

El mantenimiento de esta clase de equipos requiere de ciertos conocimientos en administración de sistemas, particularmente en el sistema operativo Linux, así como otros recursos de soporte para que la experiencia de los usuarios sea satisfactoria y facilite, de manera fluida,  la realización de las tareas requeridas. Es de esta manera que se plantea esta bitácora para documentar el proceso realizado para implementar estas herramientas, para poder restaurar todo en caso de falla o como referencia para consultas en el futuro.

\clearpage