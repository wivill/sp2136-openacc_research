\chapter{Introducción}
La computación de alto desempeño (HPC) puede definirse como un área de aplicación de la computación donde se busca la resolución de problemas muy complejos a través de la paralelización y la distribución de datos, tareas y algoritmos, todo esto a través de bibliotecas y tecnologías que optimizan este proceso.

Cada problema suele tener una serie de retos y necesidades específicas, lo cual explica la gran diversidad de herramientas que existe para poder resolverlos con mayor eficiencia, rapidez, y aprovechando los cada vez más potentes recursos computacionales disponibles.

Una de esas herramientas de paralelización es OpenACC (Open ACCelerators), el cual es un modelo de programación basado en directivas con una sintaxis e implementación en el código similar a OpenMP, con la capacidad adicional de usar aceleradores como GPUs. Este estándar fue desarrollado en conjunto por múltiples empresas como Nvidia, Cray, CAPS y PGI, así como también con la colaboración de varias universidades e instituciones académicas \cite{parallel_openacc}.

Este documento contiene una pequeña introducción sobre el estándar, cómo se implementan los pragmas a nivel de código en C/C++, cómo compilar código usando el compilador PGI disponible en el clúster Kabré y  el compilador de Nvidia nvcc. Este documento no incluye ejemplos en Fortran, aunque OpenACC si tiene una implementación compatible con ese lenguaje de programación.

\clearpage