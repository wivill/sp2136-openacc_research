\chapter{Instalación y utilización de WRF y sus diferentes módulos}
WRF es un software para simulación climática ampliamente utilizado para realizar el pronóstico del tiempo, acoplando diferentes modelos físicos, químicos y matemáticos \cite{wrf}. Este es un software sumamente complejo de instalar y utilizar, y requiere de múltiples bibliotecas. De igual forma cuenta con mucho software adicional para poder mejorar la visualización y la calidad de las simulaciones.

\section{Bibliotecas y herramientas necesarias para su utilización}
WRF como tal tiene como requisitos, bibliotecas tales como hdf y netcdf, así como herramientas tales como mpich. Inicialmente procederemos a descargar y compilar este software de acuerdo con las especificaciones recomendadas para correr WRF \cite{wrfguide}. Cada biblioteca se compilará con gcc e icc para manejar ambas versiones de WRF en el clúster. 

Primero, verificamos que los compiladores instalados localmente o como módulos funcionen adecuadamente:

\begin{lstlisting}
which gfortran
which cpp
which gcc
\end{lstlisting}

Si no se encuentra alguno de los compiladores, hacemos lo siguiente:

\begin{lstlisting}
yum group install "Development Tool"
\end{lstlisting}

Esto instalará las herramientas mínimas necesarias para que se pueda compilar WRF. También es posible ejecutar pruebas con los compiladores instalados para verificar que en verdad soportan el programa, seǵun se muestra en las pruebas de ambiente de la guía \url{http://www2.mmm.ucar.edu/wrf/OnLineTutorial/compilation_tutorial.php#STEP1}. Debemos verificar además que las siguientes herramientas estén instaladas:

\begin{lstlisting}
yum install ar awk cat cd cp cut expr file grep gzip head hostname ln ls make mkdir mv nm printf rm sed sleep sort tar touch tr uname wc which m4
\end{lstlisting}

\subsection{Compilación de bibliotecas con PGI Compiler}

\begin{lstlisting}
export DIR=/root/WRF_PGI/LIBRARIES
export CC=pgcc
export CXX=pg++
export FC=pgfortran
export FCFLAGS="-m64"
export FFLAGS="-m64"
export F77=pgfortran
module load pgi
cd /root/WRF_PGI
cd libpng-1.6.29/
CPP=cpp ./configure --prefix=$DIR/grib2
make && make install
cd ../jasper-1.900.1
./configure --prefix=$DIR/grib2
make && make install
cd ../jpeg-6b
mkdir -p /root/WRF_PGI/LIBRARIES/grib2/man/man1
./configure --prefix=$DIR/grib2
make && make install
cd ../zlib-1.2.11
./configure --prefix=$DIR/grib2
make && make install
cd ../hdf5-1.10.0-patch1
CPP=cpp ./configure --prefix=$DIR/hdf5 --with-zlib=$DIR/grib2 --disable-shared --enable-fortran
make && make install
cd ../mpich
CPP=cpp ./configure --prefix=$DIR/mpich
make && make install
cd ../netcdf-4.1.3
CPP=cpp ./configure --prefix=$DIR/netcdf --disable-dap --disable-netcdf-4 --disable-shared
\end{lstlisting}

\subsection{Módulos compilados con GCC}

\subsection{Módulos compilados con ICC}
\lstinputlisting{szip-2.1-icc.sh}
\lstinputlisting{hdf5-1.10.0-wrf.sh}
\lstinputlisting{jasper-2.0.12-icc.sh}
\lstinputlisting{libpng-1.6.29-icc.sh}
\lstinputlisting{netcdf-4.1.3-wrf.sh}
\lstinputlisting{zlib-1.2.11-icc.sh}

